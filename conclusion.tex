\section{Conclusion}
\label{sec:conclusion}
Enzyme demonstrates that it is feasible to perform efficient AD on low-level programs, opening up the door for generic language-independent AD and AD after optimization. This transforms the existing workflow machine learning researchers use to bring ML to foreign code. Instead of rewriting foreign code for machine learning, they can automatically synthesize fast gradients! This opens up the door to apply machine learning to a vast array of new use cases without the substantial effort that a rewrite or new DSL requires.

\paragraph{Broader Impact}
By reducing the amount of work necessary to apply ML to new domains, Enzyme has a generally positive impact. This advances various scientific problem domains with all the positives and negatives that come with it.

\paragraph{Future Work}
Building Enzyme as part of the LLVM compiler infrastructure opens the door for many avenues for future research. Being written as part of a framework for writing optimization, it would be interesting to explore AD-specific optimizations that might not otherwise be handled by a general optimizer. Further, one could apply one of LLVM's existing GPU or parallel code generator  on programs generated by Enzyme~\cite{grosser2011polly, gysi2020domain}. As LLVM can represent GPU~\cite{rhodin2010ptx, holewinski2011ptx} and CPU-parallel programs~\cite{tapir, doerfert2018compiler}, one could extend Enzyme to create gradients for parallel programs. One could also explore the graph problem of determine whether a value should be cached or recomputed beyond a simple heuristic. Enzyme could also be extended to support forward-mode AD, checkpointing, and mixed-mode AD~\cite{mmarxiv}. As shown by the Julia work, Enzyme opens up opportunities for cross-language AD. We would also love to work with the community to assist in porting various physics engines or other codebases to machine learning frameworks.


 
 %efficiently bring new tools 


% \todo{In order to provide a balanced perspective, authors are required to include a statement of the potential broader impact of their work, including its ethical aspects and future societal consequences. Authors should take care to discuss both positive and negative outcomes}